\PassOptionsToPackage{svgnames}{xcolor}
\documentclass[t, english]{beamer}
\usepackage[utf8]{inputenc}
\usepackage[T1]{fontenc}
\usepackage{tikz}
\usepackage{LobsterTwo}
\usepackage[default]{comfortaa}
\usepackage{pgfpages}
\usepackage{listings}
\usepackage{appendixnumberbeamer}

\lstset{
  basicstyle=\ttfamily\scriptsize\color{white},
  showstringspaces=false,
  commentstyle=\color{red},
  keywordstyle=\color{blue},
  language=bash,
  backgroundcolor=\color{black},
}

%\pgfpagesuselayout{8 on 1}[a4paper, border shrink=5mm]
%\setbeameroption{show notes on second screen=bottom}
\setbeameroption{hide notes}
%\setbeameroption{show notes}
\setbeamertemplate{note page}[plain]
\beamertemplatenavigationsymbolsempty

\synctex=1

\hypersetup{pdfpagemode=UseNone} % don't show bookmarks on initial view
\usetheme{default}
%\usefonttheme{professionalfonts}


\setbeamertemplate{footline}{%
  \raisebox{5pt}{%
    \makebox[\paperwidth]{%
      \hfill\makebox[25pt]{%
        \scriptsize\insertframenumber/\inserttotalframenumber%
      }%
    }%
  }\hspace*{5pt}%
}

% http://www.texample.net/tikz/examples/hand-drawn-lines/
\usetikzlibrary{calc,decorations.pathmorphing,patterns}
\pgfdeclaredecoration{penciline}{initial}{
    \state{initial}[width=+\pgfdecoratedinputsegmentremainingdistance,
    auto corner on length=1mm,]{
        \pgfpathcurveto%
        {% From
            \pgfqpoint{\pgfdecoratedinputsegmentremainingdistance}
                      {\pgfdecorationsegmentamplitude}
        }
        {%  Control 1
        \pgfmathrand
        \pgfpointadd{\pgfqpoint{\pgfdecoratedinputsegmentremainingdistance}{0pt}}
                    {\pgfqpoint{-\pgfdecorationsegmentaspect
                     \pgfdecoratedinputsegmentremainingdistance}%
                               {\pgfmathresult\pgfdecorationsegmentamplitude}
                    }
        }
        {%TO 
        \pgfpointadd{\pgfpointdecoratedinputsegmentlast}{\pgfpoint{1pt}{1pt}}
        }
    }
    \state{final}{}
}
\tikzset{
  photo/.style= {
    rectangle,
    line width=8pt,
    draw=DarkGreen,
    decorate,
    decoration={penciline},
    inner sep=0pt
  }
}
\input Konanur.fd
\newcommand*\initfamily{\usefont{U}{Konanur}{xl}{n}}

% http://tex.stackexchange.com/questions/16964/block-quote-with-big-quotation-marks
\newcommand*\quotesize{60}
\newcommand*{\openquote}
   {\tikz[remember picture,overlay,xshift=-4ex,yshift=-1ex]
   \node (OQ) {\fontsize{\quotesize}{\quotesize}\selectfont``};\kern0pt}
\newcommand*{\closequote}
  {\tikz[remember picture,overlay,xshift=4ex,yshift=-4ex]
   \node (CQ) {\fontsize{\quotesize}{\quotesize}\selectfont''};}
\newenvironment{shadequote}[4]%
{%
\begin{minipage}{.25\textwidth}
  \noindent
  \includegraphics[width=\textwidth]{#4}\\
  {\scriptsize\bfseries\itshape #1}
  \\
  {\tiny #2 \newline at #3}
\end{minipage}
\hfill
\begin{minipage}[c]{.7\textwidth}
\LobsterTwo
%\begin{cursive}
\begin{quote}\openquote{}
}%
{%
\hfill\closequote{}\end{quote}
%\end{cursive}
\end{minipage}
}

\begin{document}

\title{Etude et réalisation du serveur de la nouvelle plate-forme CosyVerif}
\author{Idrissa SOKHONA}
\date{29 Septembre 2014}

\LobsterTwo

\begin{frame}
\begin{center}
\par
\textsf{Soutenance de stage}

\par
\Huge Etude et réalisation du serveur de la nouvelle plate-forme CosyVerif

\par
\normalsize
\textsf{Idrissa SOKHONA}

\par
\textsf{idrissa.sokhona@etu.upmc.fr}

\par
\textsf{Université Pierre et Marie Curie,}

\par
\textsf{Master informatique,}

\par
\textsf{Spécialité SAR, Parcours SRETR}

\begin{tikzpicture}
    \node
      [anchor=north]
      at (1,3)
      {\includegraphics[height=2cm]{img/lipn}};
    \node
      [anchor=north]
      at (6.5,3)
      {\includegraphics[height=2cm]{img/lsv}};
  \end{tikzpicture}
\end{center}
\end{frame}

\begin{frame}[c]
  \frametitle{Partners}
  
  \begin{minipage}{.7\textwidth}
     \begin{itemize}
     \item <1->Qu'est ce que c'est ?,
     \item <2->Pourquoi ?,
     \item <3->Qui développe ?
     \item <4->Problèmes !
     \end{itemize}
  \end{minipage}
  
  \begin{minipage}{.25\textwidth}
  \centering
  \begin{tikzpicture}
    \node
      [anchor=north]
      at (0,3)
      {\includegraphics[height=2cm]{img/lip6}};
    \node
      [anchor=north]
      at (3,3)
      {\includegraphics[height=2cm]{img/lipn}};
    \node
      [anchor=north]
      at (6.5,3)
      {\includegraphics[height=2cm]{img/lsv}};
      {\small Pending};
  \end{tikzpicture}
  \end{minipage}
  
  \begin{minipage}{.7\textwidth}
     \begin{itemize}
     \item Partage d'outils,
     \item comparer et soutenir des études de cas industriels,
     \item promouvoir la pratique de vérification formelle.
     \end{itemize}
  \end{minipage}
 \end{frame}

\begin{frame}[plain,c]
\centering
{\Huge\LobsterTwo Qu'est ce que c'est ?}
\end{frame}

\begin{frame}[c]
  \begin{minipage}{.25\textwidth}
    \noindent
    \includegraphics[width=\textwidth]{img/cosyverif}\\
    \raggedleft\emph{\small\normalfont\url{cosyverif.org}}
  \end{minipage}
  \hfill
  \begin{minipage}[c]{.7\textwidth}
    \begin{quote}\openquote{}
    \LobsterTwo
       CosyVerif est un environnement logiciel dont le but est la spécification
       et la vérification formelle des systèmes dynamiques.
    \hfill
    \closequote{}\end{quote}

    \begin{itemize}
     \item Différents formalismes,
     \item interface graphique pour chaque formalisme,
     \item outils de vérification,
     \item outils de développeurs.
     \end{itemize}
  \end{minipage}
\end{frame}

\begin{frame}[plain,c]
\centering
{\Huge\LobsterTwo Pourquoi ?}
\end{frame}

\begin{frame}[c]
  \frametitle{Partners}
  \begin{minipage}{.25\textwidth}
  \centering
  \begin{tikzpicture}
    \node
      [anchor=north]
      at (0,3)
      {\includegraphics[height=2cm]{img/lip6}};
    \node
      [anchor=north]
      at (3,3)
      {\includegraphics[height=2cm]{img/lipn}};
    \node
      [anchor=north]
      at (6.5,3)
      {\includegraphics[height=2cm]{img/lsv}};
      {\small Pending};
  \end{tikzpicture}
  \end{minipage}
  
  \begin{minipage}{.7\textwidth}
     \begin{itemize}
     \item Partage d'outils,
     \item comparer et soutenir des études de cas industriels,
     \item promouvoir la pratique de vérification formelle.
     \end{itemize}
  \end{minipage}
 \end{frame}
  
\begin{frame}[plain,c]
\centering
{\Huge\LobsterTwo Qui développe ?}
\begin{center}
\begin{itemize}
     \item Ingénieurs,
     \item Stagiaires,
\end{itemize}
\end{center}
\end{frame}

\begin{frame}[plain,c]
\centering
{\Huge\LobsterTwo Des problèmes !}
\end{frame}


 \begin{frame}[c]
  \frametitle{Objectifs }
   
\uncover <1->{fsfsfsfsfsfssfs}

\uncover <2->{fsfsfsfsfsfssfs}

\uncover <3->{fsfsfsfsfsfssfs}


 \end{frame}






\end{document}
